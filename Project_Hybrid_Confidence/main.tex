\documentclass[10pt,onecolumn]{IEEEtran}
\usepackage{diagbox}
\usepackage{graphicx}
\usepackage{amsmath}
\usepackage{amsfonts}
\usepackage{algpseudocode}
% \usepackage{algorithm}
\usepackage{subfigure}
\usepackage{tikz}
\usepackage{epsfig}
\usepackage{cite}
\usepackage[linesnumbered,ruled]{algorithm2e}

\newtheorem{theorem}{Theorem}
\newtheorem{proposition}{Proposition}
\newtheorem{definition}{Definition}
\newtheorem{lemma}{Lemma}
\newtheorem{corollary}{Corollary}
\newtheorem{example}{Example}


\renewcommand{\ae}[1]{{\color{red}{#1}}}
\newcommand{\my}[1]{{\color{blue}{#1}}}
\newcommand{\old}[1]{{\color{green}{#1}}}
\newcommand{\pur}[1]{{\color{purple}{#1}}}




\begin{document}     
\title{Hybrid Methods: Black and White Box Models in Safety-Critical Domain}
\maketitle

\section{Desired Properties}
\begin{enumerate}
  \item  High trackability: If a false prediction occurs, the reason  for this failure can be tracked and fixed.
  \item High interpretability: The human experts could understand the model.
  \item Awareness of its limitations.  Black models usually performs well "locally" where there is a lot of training data. Thus, if the input is within the space that the model is not welled trained, the model must be aware of it.
\end{enumerate}


\section{Conformal Prediction}
In this section, we present a brief introduction about the  theory of conformal prediction~\cite{Vovk:2005}.  The conformal prediction framework was firstly proposed by Vovk et al.~\cite{Vovk:2005,vovk2009}, and since then, a lot of extensions~(e.g., \cite{Jing2013,Jing2014}) have been proposed. Consider regression data $Z^1,Z^2,\ldots,Z^n$, where each $Z^i=(X^i,Y^i)$is a random variable in $\mathbb R^d\times \mathbb R$, comprised of a response variable $Y^i$ and a d-dimensional vector of features $(X^i_1,X^i_2,\ldots,X_d^i)$. Suppose the regression function is   $$\mu(x)=\mathbb E(Y |X =x),~~x\in \mathbb R^d.$$ The conformal prediction uses past experience (e.g., $Z^1,Z^2,\ldots, Z^n$) to form prediction interval $\Gamma^{\epsilon}(X^{n+1})$ with precise levels of confidence $1-\epsilon$ in new predictions. 


The conformal prediction intervals are guaranteed to deliver proper \emph{finite-sample coverage} on the assumption that  $Z^1,Z^2,\ldots,Z^n$ are independent and identically distributed, with no knowledge of the data distribution and the regression function $\mu(x)$, i.e., 
\begin{equation}
  \mathbb P(Y\in \Gamma^{\epsilon}(X^{n+1}))\geq 1-\epsilon
\end{equation}
Let $\mathcal I$ denotes a sub set of index of regression samples and $R_{y,i}=|Y^i-\mu(X^i)|$ for $i\in \mathcal I$ denotes the fitted residual of the regression data. We define \emph{nonconformity measure} as $$A(\mathcal I,Z^i)=R_{y,i}$$  to represent how sample $Z^i$ is different from the examples in $\mathcal I$.  Given a nominal miscoverage level $\epsilon$, then 
\begin{align*}
&\Gamma^{\epsilon}(X)=\left[\hat \mu(X)-d,\hat \mu(x)+d\right]\\,\mbox{ where } d= &\mbox{ the kth smallest value in } \{R_{y,i}:i\in \mathcal I\}
\mbox{ and }k=\lceil  (|\mathcal I|+1)\times (1-\epsilon)  \rceil
\end{align*}
The details of the conformal prediction algorithm is presented in Algorithm~


\begin{algorithm}[H]
\DontPrintSemicolon
\SetAlgoLined
% \KwResult{Write here the result}
\SetKwInOut{Input}{Input}\SetKwInOut{Output}{Output}
\Input{ Data $Z^1,Z^2,\ldots, Z^n$, miscoverage level $\epsilon$, regression fit algorithm $\mathcal A$}
\Output{Prediction band}
\BlankLine
Randomly split $\{1,2,\ldots,n\}$ into two subsets $\mathcal I_1$ and $\mathcal I_2$\;
$\hat \mu=\mathcal A(\{X^i,Y^i:i\in \mathcal I_1 \})$\;
$R_{y,i}=|Y^i-\hat \mu(X^i)|~i\in \mathcal I_2$\;
$d= \mbox{ the kth smallest value in } \{R_{y,i}:i\in \mathcal I\}
\mbox{ and }k=\lceil  (|\mathcal I|+1)\times (1-\epsilon)  \rceil$\;
\Return $\left[\hat \mu(X)-d,\hat \mu(x)+d\right]$
% \While{While condition}{
%     instructions\;
%     \eIf{condition}{
%         instructions1\;
%         instructions2\;
%     }{
%         instructions3\;
%     }
% }
 
\caption{Conformal Prediction}
\end{algorithm}







\bibliographystyle{IEEEtran}
\bibliography{Refs}


\end{document}
